




\newcommand{\course}[1]{\newcommand{\courseloc}{#1}}
\newcommand{\me}[1]{\newcommand{\mynameloc}{#1}}


% Basic Packages 
\usepackage{amsmath, amssymb, amsfonts, mathtools, amsthm}
\usepackage{mathrsfs}
\usepackage{booktabs, emptypage, subcaption, multicol}
\usepackage{cancel, braket, bm, lipsum, systeme}
\usepackage{hyperref}
\usepackage[T1]{fontenc}
\usepackage{graphicx}
\usepackage{import}
\usepackage{transparent}
\usepackage{float}
\usepackage[margin=2.8cm,]{geometry}
\usepackage[shortlabels]{enumitem}
\usepackage[dvipsnames]{xcolor}
\usepackage{stmaryrd}




% figure support
\usepackage{import}
\usepackage{xifthen}
\pdfminorversion=7
\usepackage{pdfpages}
\usepackage{transparent}

\newcommand{\incfig}[2]{%
    \def\svgwidth{#1\columnwidth}
    \import{./figures/}{#2.pdf_tex}
}

\graphicspath{{./figures/}}
%% http://tex.stackexchange.com/questions/76273/multiple-pdfs-with-page-group-included-in-a-single-page-warning
\pdfsuppresswarningpagegroup=1

\newcommand{\nchapter}[2]{%
    \setcounter{chapter}{#1}%
    \addtocounter{chapter}{-1}%
    \chapter{#2}
}

\newcommand{\nsection}[3]{%
    \setcounter{chapter}{#1}%
    \setcounter{section}{#2}%
    \addtocounter{section}{-1}%
    \section{#3}
}%

\usepackage{listings}
\lstset
{
    language=[LaTeX]TeX,
    breaklines=true,
    basicstyle=\tt\scriptsize,
    keywordstyle=\color{blue},
    identifierstyle=\color{black},
}





% Misc.
\hypersetup{
    colorlinks,
    linkcolor={black},
    citecolor={black},
    urlcolor={blue!80!black},
}




% New Commands


\newcommand\N{\ensuremath{\mathbb{N}}}
\newcommand\R{\ensuremath{\mathbb{R}}}
\newcommand\Z{\ensuremath{\mathbb{Z}}}
\renewcommand\O{\ensuremath{\emptyset}}
\newcommand\Q{\ensuremath{\mathbb{Q}}}
\newcommand\C{\ensuremath{\mathbb{C}}}
\DeclareMathOperator{\sgn}{sgn}
\let\svlim\lim\def\lim{\svlim\limits}
\let\implies\Rightarrow
\let\impliedby\Leftarrow
\let\iff\Leftrightarrow
\let\epsilon\varepsilon
\newcommand\contra{\scalebox{1.1}{$\lightning$}}



% Advanced 


% correct
\definecolor{correct}{HTML}{009900}
\newcommand\correct[2]{\ensuremath{\:}{\color{red}{#1}}\ensuremath{\to }{\color{correct}{#2}}\ensuremath{\:}}
\newcommand\green[1]{{\color{correct}{#1}}}



% horizontal rule
\newcommand\hr{
    \noindent\rule[0.5ex]{\linewidth}{0.5pt}
}


% hide parts
\newcommand\hide[1]{}



% si unitx
\usepackage{siunitx}
\sisetup{locale = FR}
% \renewcommand\vec[1]{\mathbf{#1}}
\newcommand\mat[1]{\mathbf{#1}}


% tikz
\usepackage{tikz}
\usepackage{tikz-cd}
\usepackage{tikzsymbols} % for symbols such as smiley
\usetikzlibrary{intersections, angles, quotes, calc, positioning}
\usetikzlibrary{arrows.meta}
\usepackage{pgfplots}
\pgfplotsset{compat=1.13}


\tikzset{
    force/.style={thick, {Circle[length=2pt]}-stealth, shorten <=-1pt}
}


% theorems
\usepackage{thmtools}
\usepackage[framemethod=TikZ]{mdframed}
% \mdfsetup{skipabove=1em,skipbelow=0em, innertopmargin=5pt, innerbottommargin=6pt}


\theoremstyle{definition}

\makeatletter


\@ifclasswith{report}{nocolor}{
    \declaretheoremstyle[headfont=\bfseries\sffamily, bodyfont=\normalfont, mdframed={ nobreak } ]{thmgreenbox}
    \declaretheoremstyle[headfont=\bfseries\sffamily, bodyfont=\normalfont, mdframed={ nobreak } ]{thmredbox}
    \declaretheoremstyle[headfont=\bfseries\sffamily, bodyfont=\normalfont]{thmbluebox}
    \declaretheoremstyle[headfont=\bfseries\sffamily, bodyfont=\normalfont]{thmblueline}
    \declaretheoremstyle[headfont=\bfseries\sffamily, bodyfont=\normalfont, numbered=no, mdframed={ rightline=false, topline=false, bottomline=false, }, qed=\qedsymbol ]{thmproofbox}
    \declaretheoremstyle[headfont=\bfseries\sffamily, bodyfont=\normalfont, numbered=no, mdframed={ nobreak, rightline=false, topline=false, bottomline=false } ]{thmexplanationbox}
    \AtEndEnvironment{eg}{\null\hfill$\diamond$}%
}{
    \declaretheoremstyle[
        headfont=\bfseries\sffamily\color{ForestGreen!70!black}, bodyfont=\normalfont,
        mdframed={
            linewidth=2pt,
            rightline=false, topline=false, bottomline=false,
            linecolor=ForestGreen, backgroundcolor=ForestGreen!5,
        }
    ]{thmgreenbox}

    \declaretheoremstyle[
        headfont=\bfseries\sffamily\color{NavyBlue!70!black}, bodyfont=\normalfont,
        mdframed={
            linewidth=2pt,
            rightline=false, topline=false, bottomline=false,
            linecolor=NavyBlue, backgroundcolor=NavyBlue!5,
        }
    ]{thmbluebox}

    \declaretheoremstyle[
        headfont=\bfseries\sffamily\color{NavyBlue!70!black}, bodyfont=\normalfont,
        mdframed={
            linewidth=2pt,
            rightline=false, topline=false, bottomline=false,
            linecolor=NavyBlue
        }
    ]{thmblueline}

    \declaretheoremstyle[
        headfont=\bfseries\sffamily\color{RawSienna!70!black}, bodyfont=\normalfont,
        mdframed={
            linewidth=2pt,
            rightline=false, topline=false, bottomline=false,
            linecolor=RawSienna, backgroundcolor=RawSienna!5,
        }
    ]{thmredbox}

    \declaretheoremstyle[
        headfont=\bfseries\sffamily\color{RawSienna!70!black}, bodyfont=\normalfont,
        numbered=no,
        mdframed={
            linewidth=2pt,
            rightline=false, topline=false, bottomline=false,
            linecolor=RawSienna, backgroundcolor=RawSienna!1,
        },
        qed=\qedsymbol
    ]{thmproofbox}

    \declaretheoremstyle[
        headfont=\bfseries\sffamily\color{NavyBlue!70!black}, bodyfont=\normalfont,
        numbered=no,
        mdframed={
            linewidth=2pt,
            rightline=false, topline=false, bottomline=false,
            linecolor=NavyBlue, backgroundcolor=NavyBlue!1,
        },
    ]{thmexplanationbox}
}


\declaretheorem[style=thmgreenbox, name=Definition]{definition}
\declaretheorem[style=thmbluebox, numbered=yes, name=Example]{eg}
\declaretheorem[style=thmbluebox, numbered=yes, name=Exercise]{ex}
\declaretheorem[style=thmredbox, name=Proposition]{prop}
\declaretheorem[style=thmredbox, name=Theorem]{theorem}
\declaretheorem[style=thmredbox, name=Lemma]{lemma}
\declaretheorem[style=thmredbox, numbered=yes, name=Corollary]{corollary}
\declaretheorem[style=thmredbox, name=Claim]{claim}
\declaretheorem[style=thmredbox, name={Take away}]{takeaway}

\@ifclasswith{report}{nocolor}{
    \declaretheorem[style=thmproofbox, name=Proof]{replacementproof}
    \declaretheorem[style=thmexplanationbox, name=Proof]{explanation}
    \renewenvironment{proof}[1][\proofname]{\begin{replacementproof}}{\end{replacementproof}}
}{
    \declaretheorem[style=thmproofbox, name=Proof]{replacementproof}
    \renewenvironment{proof}[1][\proofname]{\vspace{-10pt}\begin{replacementproof}}{\end{replacementproof}}

    \declaretheorem[style=thmexplanationbox, name=Proof]{tmpexplanation}
    \newenvironment{explanation}[1][]{\vspace{-10pt}\begin{tmpexplanation}}{\end{tmpexplanation}}
}

\makeatother


\declaretheorem[style=thmblueline, numbered=yes, name=Remark]{remark}
\declaretheorem[style=thmblueline, numbered=yes, name=Note]{note}

\newtheorem*{uovt}{UOVT}
\newtheorem*{notation}{Notation}
\newtheorem*{previouslyseen}{As previously seen}
\newtheorem*{problem}{Problem}
\newtheorem*{observe}{Observe}
\newtheorem*{property}{Property}
\newtheorem*{intuition}{Intuition}


\usepackage{etoolbox}
\AtEndEnvironment{vb}{\null\hfill$\diamond$}%
\AtEndEnvironment{intermezzo}{\null\hfill$\diamond$}%
% \AtEndEnvironment{opmerking}{\null\hfill$\diamond$}%

% http://tex.stackexchange.com/questions/22119/how-can-i-change-the-spacing-before-theorems-with-amsthm
% \def\thm@space@setup{%
%   \thm@preskip=\parskip \thm@postskip=0pt
% }

\newcommand{\oefening}[1]{%
    \def\@oefening{#1}%
    \subsection*{Oefening #1}
}

\newcommand{\suboefening}[1]{%
    \subsubsection*{Oefening \@oefening.#1}
}

\newcommand{\exercise}[1]{%
    \def\@exercise{#1}%
    \subsection*{Exercise #1}
}

\newcommand{\subexercise}[1]{%
    \subsubsection*{Exercise \@exercise.#1}
}


\usepackage{xifthen}

% \def\testdateparts#1{\dateparts#1\relax}
% \def\dateparts#1 #2 #3 #4 #5\relax{
%     \marginpar{\small\textsf{\mbox{#1 #2 #3 #5}}}
% }

% \def\@lesson{}%
% \newcommand{\lesson}[3]{
%     \ifthenelse{\isempty{#3}}{%
%         \def\@lesson{Lecture #1}%
%     }{%
%         \def\@lesson{Lecture #1: #3}%
%     }%
%     \subsection*{\@lesson}
%     \testdateparts{#2}
% }
%
% Lecture command for course organization
%\newcommand{\lecture}[3]{%
 %   \hrule\medskip
  %  {\bf Lecture #1. #2}
   % \medskip
    %{\it #3}
    %\medskip\hrule\bigskip
%}
\newcommand{\lecture}[3]{%
    \newpage  % Start each lecture on new page
    \addcontentsline{toc}{chapter}{Lecture #1: #2}
    \hrule\medskip
    {\Large\bf Lecture #1. #2}
    \medskip
    {\it Date: #3}
    \medskip\hrule\bigskip 
    \vspace{1em}
}
% Chapter title
\usepackage{titlesec}
\titleformat{\chapter}[frame]
  {\normalfont}
  {\filright
   \footnotesize
   \enspace Lecture \arabic{chapter}.\enspace}
  {8pt}
  {\Large\bfseries\filcenter}
\usepackage[dotinlabels]{titletoc}
\titlecontents{chapter}[1.5em]{}{\contentslabel{2.3em}}{\hspace*{-2.3em}}{\hfill\contentspage}
\titlespacing*{\chapter} {0pt}{0pt}{40pt}     % this alters "before" spacing (the second length argument) to 0

% \renewcommand\date[1]{\marginpar{#1}}


% % fancy headers
% \usepackage{fancyhdr}
% \pagestyle{fancy}

% % \fancypagestyle{header}{%
% % \fancyhead[LE,RO]{Mubtasim Fuad}
% \fancyhead[R]{Lecture \thechapter}
% \fancyhead[L]{}
% \fancyfoot[C]{\thepage}
% % \fancyfoot[C]{\leftmark}
% % }

% Style for page headers and footers
\usepackage{fancyhdr}
\fancypagestyle{head}{
  \fancyhf{}     % clear all header and footer fields
  \lhead{\courseloc}
  \rhead{Lecture \thechapter}
  \cfoot{\thepage}    % Use \pageref{LastPage} instead if you want to add the link
  \renewcommand{\headrulewidth}{0.5pt}
  \renewcommand{\footrulewidth}{0.5pt}
}

\fancypagestyle{plain}{
  \fancyhf{}     % clear all header and footer fields
  \rhead{\courseloc}
  \cfoot{\thepage}    % Use \pageref{LastPage} instead if you want to add the link
  \renewcommand{\headrulewidth}{0.5pt}
  \renewcommand{\footrulewidth}{0.5pt}   
}

\makeatother


% notes
% \usepackage{todonotes} % slowdown compilation speed
\usepackage{marginnote}
\let\marginpar\marginnote

% sidenotes with arrows in equations
\usepackage{witharrows}

% Some custom colorbox and sidenote environments
\usepackage{tcolorbox}

\tcbuselibrary{breakable}
\newenvironment{verbetering}{\begin{tcolorbox}[
    arc=0mm,
    colback=white,
    colframe=green!60!black,
    title=Opmerking,
    fonttitle=\sffamily,
    breakable
]}{\end{tcolorbox}}

% \newenvironment{noot}[1]{\begin{tcolorbox}[
%     arc=0mm,
%     colback=white,
%     colframe=white!60!black,
%     title=#1,
%     fonttitle=\sffamily,
%     breakable
% ]}{\end{tcolorbox}}

% Side note with custom color
\newtcolorbox{noot}[3][]
{
  arc=0mm,
  colback  = white,
  colframe = #2!50,
  coltitle = #2!20!black,   
  title    = {Side-note: #3},
  fonttitle=\sffamily,
  breakable,
  #1,
}

% https://tex.stackexchange.com/a/172480/114006
\newtcolorbox{mybox}[3][]
{
  colback  = #2!10,
  colframe = #2!25,
  coltitle = #2!20!black,  
  title    = {#3},
  #1,
}

% Side note environment with gray color and title
\newenvironment{sidenote}[1]{\begin{noot}{gray}{#1}}{\end{noot}}
% Side note environment with user-defined color and title
\newenvironment{sidenotex}[2]{\begin{noot}{#1}{#2}}{\end{noot}}

\newenvironment{myminipage}
    {
    \begin{center}
    \begin{minipage}{0.85\textwidth}
    %  \begin{mdframed}
    }
    { 
    %  \end{mdframed}
    \end{minipage}   
    \end{center}
    }




% Change paragraph spacing
\setlength{\parskip}{4pt}
% Change paragraph indent
\setlength{\parindent}{0cm}

% for creating \NewDocumentCommand
% \usepackage{xparse}


%% for setting arrows beneath equations 

% new command notate - https://tex.stackexchange.com/a/263491/114006
\usepackage[usestackEOL]{stackengine}
\usepackage{scalerel}

% \parskip \baselineskip  % it's creating problem with my theorem environment box designs
\def\svmybf#1{\rotatebox{90}{\stretchto{\{}{#1}}}
\def\svnobf#1{}
\def\rlwd{.5pt}
\newcommand\notate[4][B]{%
  \if B#1\let\myupbracefill\svmybf\else\let\myupbracefill\svnobf\fi%
  \def\useanchorwidth{T}%
  \setbox0=\hbox{$\displaystyle#2$}%
  \def\stackalignment{c}\stackunder[-6pt]{%
    \def\stackalignment{c}\stackunder[-1.5pt]{%
      \stackunder[2pt]{\strut $\displaystyle#2$}{\myupbracefill{\wd0}}}{%
    \rule{\rlwd}{#3\baselineskip}}}{%
  \strut\kern9pt$\rightarrow$\smash{\rlap{$~\displaystyle#4$}}}%
}

% new command indicate - https://tex.stackexchange.com/a/15742/114006

% trees
\usepackage[linguistics]{forest}

% some packages from my other templates 
\usepackage{array} % for tables
\usepackage[export]{adjustbox}
% \usepackage{wrapfig} % not properly fitting here
\usepackage{multirow}
\usepackage{tabularx}
\usepackage{extarrows}  % https://tex.stackexchange.com/a/407628/114006

% bibliography
\usepackage[
    backend=biber, 
    backref=true, 
    style=numeric, 
    sortcites=true, 
    sorting=none, 
    defernumbers=true
]{biblatex}       % bibliography  % https://www.overleaf.com/learn/latex/bibliography_management_with_biblatex
\usepackage{xurl}                % handling the urls in bib file and it should be loaded after loading biblatex 

% Allow page breaks inside equation environments of amsmath package
\allowdisplaybreaks

% To cancel terms in equations with diagonal arrows
\usepackage{cancel} % https://tex.stackexchange.com/a/75530/114006
\usepackage{bookmark}
% To write tensors with mixed indices
\usepackage{tensor} % https://tex.stackexchange.com/a/25976/114006
\usepackage{derivative} % https://tex.stackexchange.com/a/14822/114006 % https://tex.stackexchange.com/a/501336/114006
\usepackage{annotate-equations} % https://github.com/st--/annotate-equations

% To create dashed box in equations - https://tex.stackexchange.com/a/285489/114006
\usepackage{dashbox}
\newcommand\dboxed[1]{\dbox{\ensuremath{#1}}}

% To add Left-aligned texts in equations (doesn't work in equation environment) - https://tex.stackexchange.com/a/60168/114006 
\makeatletter
\newif\if@gather@prefix 
\preto\place@tag@gather{% 
  \if@gather@prefix\iftagsleft@ 
    \kern-\gdisplaywidth@ 
    \rlap{\gather@prefix}% 
    \kern\gdisplaywidth@ 
  \fi\fi 
} 
\appto\place@tag@gather{% 
  \if@gather@prefix\iftagsleft@\else 
    \kern-\displaywidth 
    \rlap{\gather@prefix}% 
    \kern\displaywidth 
  \fi\fi 
  \global\@gather@prefixfalse 
} 
\preto\place@tag{% 
  \if@gather@prefix\iftagsleft@ 
    \kern-\gdisplaywidth@ 
    \rlap{\gather@prefix}% 
    \kern\displaywidth@ 
  \fi\fi 
} 
\appto\place@tag{% 
  \if@gather@prefix\iftagsleft@\else 
    \kern-\displaywidth 
    \rlap{\gather@prefix}% 
    \kern\displaywidth 
  \fi\fi 
  \global\@gather@prefixfalse 
} 
\def\math@cr@@@align{%
  \ifst@rred\nonumber\fi
  \if@eqnsw \global\tag@true \fi
  \global\advance\row@\@ne
  \add@amps\maxfields@
  \omit
  \kern-\alignsep@
  \if@gather@prefix\tag@true\fi
  \iftag@
    \setboxz@h{\@lign\strut@{\make@display@tag}}%
    \place@tag
  \fi
  \ifst@rred\else\global\@eqnswtrue\fi
  \global\lineht@\z@
  \cr
}
\newcommand*{\lefttext}[1]{% 
  \ifmeasuring@\else
  \gdef\gather@prefix{#1}% 
  \global\@gather@prefixtrue 
  \fi
} 
\makeatother

% \usepackage[none]{hyphenat}
\usepackage{nicematrix} 

% Command for circle around text [used chatgpt]
\newcommand{\mycir}[1]{%
    \mathchoice%
        {\mycirAux{\displaystyle}{#1}}%
        {\mycirAux{\textstyle}{#1}}%
        {\mycirAux{\scriptstyle}{#1}}%
        {\mycirAux{\scriptscriptstyle}{#1}}%
}
\newcommand{\mycirAux}[2]{%
        \tikz[baseline=(char.base)]{%
            \node[draw, circle, inner sep=1pt, font={\fontsize{8}{8}\selectfont}] (char) 
            {\ensuremath{#1{#2}}};
        }
}

% Custom enviroment for table-like matrix / cagedmatrix % https://tex.stackexchange.com/a/686197/114006

\ExplSyntaxOn
\NewDocumentEnvironment{cagedmatrix}{b}
 {
  % split the input at \\
  \seq_set_split:Nnn \l_tmpa_seq { \\ } { #1 }
  % check for a missing trailing \\
  \seq_pop_right:NN \l_tmpa_seq \l_tmpa_tl
  \tl_if_empty:NF \l_tmpa_tl { \seq_put_right:NV \l_tmpa_seq \l_tmpa_tl }
  % build the array, inserting \\ \hline between rows
  \array{|*{\value{MaxMatrixCols}}{c|}}\hline
  \seq_use:Nn \l_tmpa_seq { \\ \hline }
  % finish up
  \\ \hline
  \endarray
}{}
\ExplSyntaxOff

% Another method: using ytableau but it has issues in case of horizontal-aligning with other terms in equations.  
\usepackage{ytableau}

% Another custom commnad for tabularmatrix % https://tex.stackexchange.com/a/686232/114006
\usepackage{tabularray}
\NewDocumentEnvironment{tabularmatrix}{+b}{
    \begin{tblr}{
       hlines, vlines, columns={c},
       rowsep=0.1pt, colsep=5pt,
       }
    #1
    \end{tblr}
    }{}

% Define cagedbox command
\newcommand{\cagedbox}[1]{%
  \begin{tabularmatrix}%
    #1%
  \end{tabularmatrix}%
}

%% Change formatting of back references % https://tex.stackexchange.com/a/606518/114006
\DefineBibliographyStrings{english}{
   backrefpage={p.},
  % backrefpage={},
   backrefpages={pp.}
  % backrefpages={
}
\renewcommand*{\finentrypunct}{}
\usepackage{xpatch}

% \DeclareFieldFormat{backrefparens}{\mkbibparens{#1\addperiod}}
\DeclareFieldFormat{backrefparens}{\unskip.~\raisebox{-4pt}{\scriptsize{\mkbibparens{#1}}}}
\xpatchbibmacro{pageref}{parens}{backrefparens}{}{}







 %new things

\newcommand{\liff}{\leftrightarrow}
\newcommand{\lthen}{\rightarrow}
\newcommand{\opname}{\operatorname}
\newcommand{\surjto}{\twoheadrightarrow}
\newcommand{\injto}{\hookrightarrow}
\newcommand{\On}{\mathrm{On}} % ordinals
\DeclareMathOperator{\img}{im} % Image
\DeclareMathOperator{\Img}{Im} % Image
\DeclareMathOperator{\coker}{coker} % Cokernel
\DeclareMathOperator{\Coker}{Coker} % Cokernel
\DeclareMathOperator{\Ker}{Ker} % Kernel
\DeclareMathOperator{\rank}{rank}
\DeclareMathOperator{\Spec}{Spec} % spectrum
\DeclareMathOperator{\Tr}{Tr} % trace
\DeclareMathOperator{\pr}{pr} % projection
\DeclareMathOperator{\ext}{ext} % extension
\DeclareMathOperator{\pred}{pred} % predecessor
\DeclareMathOperator{\dom}{dom} % domain
\DeclareMathOperator{\ran}{ran} % range
\DeclareMathOperator{\Hom}{Hom} % homomorphism
\DeclareMathOperator{\Mor}{Mor} % morphisms
\DeclareMathOperator{\End}{End} % endomorphism




% \DeclareMathOperator{\GL}{GL}
\DeclareMathOperator{\im}{Im}
% \DeclareMathOperator{\Ker}{Ker}
% \DeclareMathOperator{\Hom}{Hom}
% \DeclareMathOperator{\Tr}{Tr}
\DeclareMathOperator{\Hol}{Hol}
% \DeclareMathOperator{\Aut}{Aut}
\DeclareMathOperator{\Fit}{Fitt}
% \DeclareMathOperator{\coker}{coker}
% \DeclareMathOperator{\Ext}{Ext}
% \DeclareMathOperator{\Tor}{Tor}
\DeclareMathOperator{\Der}{Der}
\DeclareMathOperator{\PDer}{PDer}

%%%%%%%%%%%%%%%%%%%%%%%%%%%%%%%%%%%%%

%% From SeniorMars lecture template

% Things Lie
\newcommand{\kb}{\mathfrak b}
\newcommand{\kg}{\mathfrak g}
\newcommand{\kh}{\mathfrak h}
\newcommand{\kn}{\mathfrak n}
\newcommand{\ku}{\mathfrak u}
\newcommand{\kz}{\mathfrak z}
\DeclareMathOperator{\Ext}{Ext} % Ext functor
\DeclareMathOperator{\Tor}{Tor} % Tor functor
\newcommand{\gl}{\opname{\mathfrak{gl}}} % frak gl group
\renewcommand{\sl}{\opname{\mathfrak{sl}}} % frak sl group chktex 6
\newcommand\Lie{\pounds} % Lie Derivative % https://tex.stackexchange.com/a/102689/114006

% More script letters etc.
\newcommand{\SA}{\mathcal A}
\newcommand{\SB}{\mathcal B}
\newcommand{\SC}{\mathcal C}
\newcommand{\SF}{\mathcal F}
\newcommand{\SG}{\mathcal G}
\newcommand{\SH}{\mathcal H}
\newcommand{\OO}{\mathcal O}

\newcommand{\SCA}{\mathscr A}
\newcommand{\SCB}{\mathscr B}
\newcommand{\SCC}{\mathscr C}
\newcommand{\SCD}{\mathscr D}
\newcommand{\SCE}{\mathscr E}
\newcommand{\SCF}{\mathscr F}
\newcommand{\SCG}{\mathscr G}
\newcommand{\SCH}{\mathscr H}

% Mathfrak primes
\newcommand{\km}{\mathfrak m}
\newcommand{\kp}{\mathfrak p}
\newcommand{\kq}{\mathfrak q}

% number sets
\newcommand{\RR}[1][]{\ensuremath{\ifstrempty{#1}{\mathbb{R}}{\mathbb{R}^{#1}}}}
\newcommand{\NN}[1][]{\ensuremath{\ifstrempty{#1}{\mathbb{N}}{\mathbb{N}^{#1}}}}
\newcommand{\ZZ}[1][]{\ensuremath{\ifstrempty{#1}{\mathbb{Z}}{\mathbb{Z}^{#1}}}}
\newcommand{\QQ}[1][]{\ensuremath{\ifstrempty{#1}{\mathbb{Q}}{\mathbb{Q}^{#1}}}}
\newcommand{\CC}[1][]{\ensuremath{\ifstrempty{#1}{\mathbb{C}}{\mathbb{C}^{#1}}}}
\newcommand{\PP}[1][]{\ensuremath{\ifstrempty{#1}{\mathbb{P}}{\mathbb{P}^{#1}}}}
\newcommand{\HH}[1][]{\ensuremath{\ifstrempty{#1}{\mathbb{H}}{\mathbb{H}^{#1}}}}
\newcommand{\FF}[1][]{\ensuremath{\ifstrempty{#1}{\mathbb{F}}{\mathbb{F}^{#1}}}}
% expected value
\newcommand{\EE}{\ensuremath{\mathbb{E}}}
\newcommand{\charin}{\text{ char }}
\DeclareMathOperator{\sign}{sign}
\DeclareMathOperator{\Aut}{Aut}
\DeclareMathOperator{\Inn}{Inn}
\DeclareMathOperator{\Syl}{Syl}
\DeclareMathOperator{\Gal}{Gal}
\DeclareMathOperator{\GL}{GL} % General linear group
\DeclareMathOperator{\SL}{SL} % Special linear group

%---------------------------------------
% BlackBoard Math Fonts :-
%---------------------------------------

%Captital Letters
\newcommand{\bbA}{\mathbb{A}}	\newcommand{\bbB}{\mathbb{B}}
\newcommand{\bbC}{\mathbb{C}}	\newcommand{\bbD}{\mathbb{D}}
\newcommand{\bbE}{\mathbb{E}}	\newcommand{\bbF}{\mathbb{F}}
\newcommand{\bbG}{\mathbb{G}}	\newcommand{\bbH}{\mathbb{H}}
\newcommand{\bbI}{\mathbb{I}}	\newcommand{\bbJ}{\mathbb{J}}
\newcommand{\bbK}{\mathbb{K}}	\newcommand{\bbL}{\mathbb{L}}
\newcommand{\bbM}{\mathbb{M}}	\newcommand{\bbN}{\mathbb{N}}
\newcommand{\bbO}{\mathbb{O}}	\newcommand{\bbP}{\mathbb{P}}
\newcommand{\bbQ}{\mathbb{Q}}	\newcommand{\bbR}{\mathbb{R}}
\newcommand{\bbS}{\mathbb{S}}	\newcommand{\bbT}{\mathbb{T}}
\newcommand{\bbU}{\mathbb{U}}	\newcommand{\bbV}{\mathbb{V}}
\newcommand{\bbW}{\mathbb{W}}	\newcommand{\bbX}{\mathbb{X}}
\newcommand{\bbY}{\mathbb{Y}}	\newcommand{\bbZ}{\mathbb{Z}}

%---------------------------------------
% MathCal Fonts :-
%---------------------------------------

%Captital Letters
\newcommand{\mcA}{\mathcal{A}}	\newcommand{\mcB}{\mathcal{B}}
\newcommand{\mcC}{\mathcal{C}}	\newcommand{\mcD}{\mathcal{D}}
\newcommand{\mcE}{\mathcal{E}}	\newcommand{\mcF}{\mathcal{F}}
\newcommand{\mcG}{\mathcal{G}}	\newcommand{\mcH}{\mathcal{H}}
\newcommand{\mcI}{\mathcal{I}}	\newcommand{\mcJ}{\mathcal{J}}
\newcommand{\mcK}{\mathcal{K}}	\newcommand{\mcL}{\mathcal{L}}
\newcommand{\mcM}{\mathcal{M}}	\newcommand{\mcN}{\mathcal{N}}
\newcommand{\mcO}{\mathcal{O}}	\newcommand{\mcP}{\mathcal{P}}
\newcommand{\mcQ}{\mathcal{Q}}	\newcommand{\mcR}{\mathcal{R}}
\newcommand{\mcS}{\mathcal{S}}	\newcommand{\mcT}{\mathcal{T}}
\newcommand{\mcU}{\mathcal{U}}	\newcommand{\mcV}{\mathcal{V}}
\newcommand{\mcW}{\mathcal{W}}	\newcommand{\mcX}{\mathcal{X}}
\newcommand{\mcY}{\mathcal{Y}}	\newcommand{\mcZ}{\mathcal{Z}}
%Small Letters
\newcommand{\mca}{\mathcal{a}}	\newcommand{\mcb}{\mathcal{b}}
\newcommand{\mcc}{\mathcal{c}}	\newcommand{\mcd}{\mathcal{d}}
\newcommand{\mce}{\mathcal{e}}	\newcommand{\mcf}{\mathcal{f}}
\newcommand{\mcg}{\mathcal{g}}	\newcommand{\mch}{\mathcal{h}}
\newcommand{\mci}{\mathcal{i}}	\newcommand{\mcj}{\mathcal{j}}
\newcommand{\mck}{\mathcal{k}}	\newcommand{\mcl}{\mathcal{l}}
\newcommand{\mcm}{\mathcal{m}}	\newcommand{\mcn}{\mathcal{n}}
\newcommand{\mco}{\mathcal{o}}	\newcommand{\mcp}{\mathcal{p}}
\newcommand{\mcq}{\mathcal{q}}	\newcommand{\mcr}{\mathcal{r}}
\newcommand{\mcs}{\mathcal{s}}	\newcommand{\mct}{\mathcal{t}}
\newcommand{\mcu}{\mathcal{u}}	\newcommand{\mcv}{\mathcal{v}}
\newcommand{\mcw}{\mathcal{w}}	\newcommand{\mcx}{\mathcal{x}}
\newcommand{\mcy}{\mathcal{y}}	\newcommand{\mcz}{\mathcal{z}}

%---------------------------------------
% Bold Math Fonts :-
%---------------------------------------

%Captital Letters
\newcommand{\bmA}{\boldsymbol{A}}	\newcommand{\bmB}{\boldsymbol{B}}
\newcommand{\bmC}{\boldsymbol{C}}	\newcommand{\bmD}{\boldsymbol{D}}
\newcommand{\bmE}{\boldsymbol{E}}	\newcommand{\bmF}{\boldsymbol{F}}
\newcommand{\bmG}{\boldsymbol{G}}	\newcommand{\bmH}{\boldsymbol{H}}
\newcommand{\bmI}{\boldsymbol{I}}	\newcommand{\bmJ}{\boldsymbol{J}}
\newcommand{\bmK}{\boldsymbol{K}}	\newcommand{\bmL}{\boldsymbol{L}}
\newcommand{\bmM}{\boldsymbol{M}}	\newcommand{\bmN}{\boldsymbol{N}}
\newcommand{\bmO}{\boldsymbol{O}}	\newcommand{\bmP}{\boldsymbol{P}}
\newcommand{\bmQ}{\boldsymbol{Q}}	\newcommand{\bmR}{\boldsymbol{R}}
\newcommand{\bmS}{\boldsymbol{S}}	\newcommand{\bmT}{\boldsymbol{T}}
\newcommand{\bmU}{\boldsymbol{U}}	\newcommand{\bmV}{\boldsymbol{V}}
\newcommand{\bmW}{\boldsymbol{W}}	\newcommand{\bmX}{\boldsymbol{X}}
\newcommand{\bmY}{\boldsymbol{Y}}	\newcommand{\bmZ}{\boldsymbol{Z}}
%Small Letters
\newcommand{\bma}{\boldsymbol{a}}	\newcommand{\bmb}{\boldsymbol{b}}
\newcommand{\bmc}{\boldsymbol{c}}	\newcommand{\bmd}{\boldsymbol{d}}
\newcommand{\bme}{\boldsymbol{e}}	\newcommand{\bmf}{\boldsymbol{f}}
\newcommand{\bmg}{\boldsymbol{g}}	\newcommand{\bmh}{\boldsymbol{h}}
\newcommand{\bmi}{\boldsymbol{i}}	\newcommand{\bmj}{\boldsymbol{j}}
\newcommand{\bmk}{\boldsymbol{k}}	\newcommand{\bml}{\boldsymbol{l}}
\newcommand{\bmm}{\boldsymbol{m}}	\newcommand{\bmn}{\boldsymbol{n}}
\newcommand{\bmo}{\boldsymbol{o}}	\newcommand{\bmp}{\boldsymbol{p}}
\newcommand{\bmq}{\boldsymbol{q}}	\newcommand{\bmr}{\boldsymbol{r}}
\newcommand{\bms}{\boldsymbol{s}}	\newcommand{\bmt}{\boldsymbol{t}}
\newcommand{\bmu}{\boldsymbol{u}}	\newcommand{\bmv}{\boldsymbol{v}}
\newcommand{\bmw}{\boldsymbol{w}}	\newcommand{\bmx}{\boldsymbol{x}}
\newcommand{\bmy}{\boldsymbol{y}}	\newcommand{\bmz}{\boldsymbol{z}}

%---------------------------------------
% Scr Math Fonts :-
%---------------------------------------

\newcommand{\sA}{{\mathscr{A}}}   \newcommand{\sB}{{\mathscr{B}}}
\newcommand{\sC}{{\mathscr{C}}}   \newcommand{\sD}{{\mathscr{D}}}
\newcommand{\sE}{{\mathscr{E}}}   \newcommand{\sF}{{\mathscr{F}}}
\newcommand{\sG}{{\mathscr{G}}}   \newcommand{\sH}{{\mathscr{H}}}
\newcommand{\sI}{{\mathscr{I}}}   \newcommand{\sJ}{{\mathscr{J}}}
\newcommand{\sK}{{\mathscr{K}}}   \newcommand{\sL}{{\mathscr{L}}}
\newcommand{\sM}{{\mathscr{M}}}   \newcommand{\sN}{{\mathscr{N}}}
\newcommand{\sO}{{\mathscr{O}}}   \newcommand{\sP}{{\mathscr{P}}}
\newcommand{\sQ}{{\mathscr{Q}}}   \newcommand{\sR}{{\mathscr{R}}}
\newcommand{\sS}{{\mathscr{S}}}   \newcommand{\sT}{{\mathscr{T}}}
\newcommand{\sU}{{\mathscr{U}}}   \newcommand{\sV}{{\mathscr{V}}}
\newcommand{\sW}{{\mathscr{W}}}   \newcommand{\sX}{{\mathscr{X}}}
\newcommand{\sY}{{\mathscr{Y}}}   \newcommand{\sZ}{{\mathscr{Z}}}


%---------------------------------------
% Math Fraktur Font
%---------------------------------------

%Captital Letters
\newcommand{\mfA}{\mathfrak{A}}	\newcommand{\mfB}{\mathfrak{B}}
\newcommand{\mfC}{\mathfrak{C}}	\newcommand{\mfD}{\mathfrak{D}}
\newcommand{\mfE}{\mathfrak{E}}	\newcommand{\mfF}{\mathfrak{F}}
\newcommand{\mfG}{\mathfrak{G}}	\newcommand{\mfH}{\mathfrak{H}}
\newcommand{\mfI}{\mathfrak{I}}	\newcommand{\mfJ}{\mathfrak{J}}
\newcommand{\mfK}{\mathfrak{K}}	\newcommand{\mfL}{\mathfrak{L}}
\newcommand{\mfM}{\mathfrak{M}}	\newcommand{\mfN}{\mathfrak{N}}
\newcommand{\mfO}{\mathfrak{O}}	\newcommand{\mfP}{\mathfrak{P}}
\newcommand{\mfQ}{\mathfrak{Q}}	\newcommand{\mfR}{\mathfrak{R}}
\newcommand{\mfS}{\mathfrak{S}}	\newcommand{\mfT}{\mathfrak{T}}
\newcommand{\mfU}{\mathfrak{U}}	\newcommand{\mfV}{\mathfrak{V}}
\newcommand{\mfW}{\mathfrak{W}}	\newcommand{\mfX}{\mathfrak{X}}
\newcommand{\mfY}{\mathfrak{Y}}	\newcommand{\mfZ}{\mathfrak{Z}}
%Small Letters
\newcommand{\mfa}{\mathfrak{a}}	\newcommand{\mfb}{\mathfrak{b}}
\newcommand{\mfc}{\mathfrak{c}}	\newcommand{\mfd}{\mathfrak{d}}
\newcommand{\mfe}{\mathfrak{e}}	\newcommand{\mff}{\mathfrak{f}}
\newcommand{\mfg}{\mathfrak{g}}	\newcommand{\mfh}{\mathfrak{h}}
\newcommand{\mfi}{\mathfrak{i}}	\newcommand{\mfj}{\mathfrak{j}}
\newcommand{\mfk}{\mathfrak{k}}	\newcommand{\mfl}{\mathfrak{l}}
\newcommand{\mfm}{\mathfrak{m}}	\newcommand{\mfn}{\mathfrak{n}}
\newcommand{\mfo}{\mathfrak{o}}	\newcommand{\mfp}{\mathfrak{p}}
\newcommand{\mfq}{\mathfrak{q}}	\newcommand{\mfr}{\mathfrak{r}}
\newcommand{\mfs}{\mathfrak{s}}	\newcommand{\mft}{\mathfrak{t}}
\newcommand{\mfu}{\mathfrak{u}}	\newcommand{\mfv}{\mathfrak{v}}
\newcommand{\mfw}{\mathfrak{w}}	\newcommand{\mfx}{\mathfrak{x}}
\newcommand{\mfy}{\mathfrak{y}}	\newcommand{\mfz}{\mathfrak{z}}

% New command for raised \chi % https://tex.stackexchange.com/a/103898/114006 
\DeclareRobustCommand{\rchi}{{\mathpalette\irchi\relax}}
\newcommand{\irchi}[2]{\raisebox{\depth}{$#1\chi$}} % inner command, used by \rchi
