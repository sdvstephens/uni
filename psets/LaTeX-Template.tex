\documentclass[12pt]{amsart}

%Below are some necessary packages for your course.
\usepackage{amsfonts,latexsym,amsthm,amssymb,amsmath,amscd,euscript,enumerate}
\usepackage{framed}
\usepackage{fullpage}
\usepackage{hyperref}
    \hypersetup{colorlinks=true,citecolor=blue,urlcolor =black,linkbordercolor={1 0 0}}

\newenvironment{statement}[1]{\smallskip\noindent\color[rgb]{1.00,0.00,0.50} {\bf #1.}}{}
\allowdisplaybreaks[1]

%Below are the theorem, definition, example, lemma, etc. body types.

\newtheorem{theorem}{Theorem}
\newtheorem*{proposition}{Proposition}
\newtheorem{lemma}[theorem]{Lemma}
\newtheorem{corollary}[theorem]{Corollary}
\newtheorem{conjecture}[theorem]{Conjecture}
\newtheorem{postulate}[theorem]{Postulate}
\theoremstyle{definition}
\newtheorem{defn}[theorem]{Definition}
\newtheorem{example}[theorem]{Example}

\theoremstyle{remark}
\newtheorem*{remark}{Remark}
\newtheorem*{notation}{Notation}
\newtheorem*{note}{Note}

% You can define new commands to make your life easier.
\newcommand{\R}{\mathbb R}
\newcommand{\C}{\mathbb C}
\newcommand{\F}{\mathbb F}
\newcommand{\Q}{\mathbb Q}
\newcommand{\Z}{\mathbb Z}
\newcommand{\N}{\mathbb N}
\def\ch{\mathrm{char}}
\def\Hom{\mathrm{Hom}}

\def\Ker{\mathrm{Ker}}
\def\Im{\mathrm{Im}}
\def\Vect{\mathrm{Vect}}
\def\rank{\mathrm{rank}}
% We can even define a new command for \newcommand!
\newcommand{\nc}{\newcommand}

% If you want a new function, use operatorname to define that function (don't use \text)
\nc{\on}{\operatorname}
\nc{\Spec}{\on{Spec}}

\title{Math 55a pset 3} % IMPORTANT: Change the problemset number as needed.
\date{\today}

\begin{document}

\maketitle

\vspace*{-0.25in}
\centerline{SDVS}
% Just so that your CA's can come knocking on your door when you don't hand in that problemset on time...
\centerline{}
\centerline{Cambridge, MA 02138}
\centerline{\href{mailto:youremailhere@college.harvard.edu}{{\tt youremailhere@college.harvard.edu}}}
\vspace*{0.15in}



\begin{statement}{1}
 Let $V$ be an $n$-dimensional vector space, and $T : V \to V$ a linear map; suppose that
$T$ is nilpotent, i.e.\ $T^N = 0$ for some $N > 0$.

  \begin{enumerate}[(a)]
    \item Show that $T^n = 0$.

    \item   Show that \( I+T \) is invertible (where \( I \) is the identity operator), and give a formula for its inverse. 
  \end{enumerate}
\end{statement}

\begin{proof}
We begin with (a). Note that this creates three cases, 
\begin{enumerate}[I.]
  \item \( n=N \), trivial;
  \item \( n>N, \implies \, T^n=T^NT^{n-N}=T^{n-N}0=0 \);
  \item \( n<N \), 
\end{enumerate}
\end{proof}

\begin{statement}{2}
 Let $V$ be a vector space over a field $k$ and $\phi:V\to V$ a
linear operator. dimensions of the kernal and image of \( T \)? What changes if \( \text{char}=2 \)?

(a) Show that for all $m\geq 1$, $\Im(\phi^{m+1})\subset \Im(\phi^m)$,
and if $\Im(\phi^{m+1})=\Im(\phi^m)$ then $\Im(\phi^n)=\Im(\phi^m)$ for all
$n\geq m$.

(b) The {\em eventual image} of $\phi$, denoted $\mathrm{evIm}(\phi)$, is
the set of vectors which can be expressed as $\phi^m(v)$ for {\em all}
$m\in\N$, i.e.\ $\mathrm{evIm}(\phi)=\bigcap_{m\geq 1} \Im(\phi^m)$.
Show that $\mathrm{evIm}(\phi)$ is an invariant subspace for $\phi$.

(c) Show that, if $V$ is finite-dimensional, then the eventual image of
$\phi$ and its generalized kernel $\mathrm{gKer}(\phi)=\{v\in V\,|\,
\exists m\in \N,\ \phi^m(v)=0\}$ coincide with the image and kernel of 
$\phi^n$ where $n=\dim V$, and
the restriction of $\phi$ to $\mathrm{evIm}(\phi)$ is surjective.

(d) Still assuming $V$ is finite-dimensional, deduce from the above results
that $V$ decomposes into the direct sum of invariant subspaces
$V=\mathrm{evIm}(\phi)\oplus \mathrm{gKer}(\phi)$, where $\phi$ is invertible
on $\mathrm{evIm}(\phi)$ and nilpotent on $\mathrm{gKer}(\phi)$.

(e) Show that, if $V$ is infinite-dimensional, then none of the 
statements in (d) need to hold: find an infinite-dimensional vector 
space 	$V$ and two linear operators $\phi,\psi:V\to V$ for which:
\begin{enumerate}
\item $\mathrm{evIm}(\phi)=\mathrm{gKer}(\phi)=V$,
the restriction of $\phi$ to $\mathrm{evIm}(\phi)$ is not injective,
and the restriction of $\phi$ to $\mathrm{gKer}(\phi)$ is not nilpotent;
\item $\mathrm{evIm}(\psi)=\mathrm{gKer}(\psi)=0$.
\end{enumerate}
\end{statement}
\begin{proof}
    
\end{proof}

\begin{statement}{3}
  
 Fix a field $k$, and consider the category $\Vect_k$ of
all vector spaces over $k$.

(a) Show that there exists a contravariant functor from the category $\Vect_k$ to
itself, which on objects takes each vector space $V$ to
its dual $V^*=\Hom(V,k)$.

(b) Recall that for each vector space $V$ we have a ``natural'' homomorphism 
$ev_V:V\to V^{**}$ taking every vector $v\in V$ to the element
$ev_V(v)$ of $V^{**}=\Hom(V^*,k)$ which maps $\ell\in V^*$ to $\ell(v)\in k$.
Show that these homomorphisms determine a natural transformation from
the identity functor to the square of the functor of part (a).
\end{statement}



\begin{statement}{4}

  \begin{enumerate}[(a)]
    \item Find a field $\F_4$ with 4 elements! Namely, denote the elements by $\{0, 1, \alpha, \beta\}$ and write out the tables for addition and multiplication in $\F_4$. 
    \item If we forget the multiplicative structure of $\F_4$ and just think of it as an abelian group
for addition, is it isomorphic to $\Z/4$ or $\Z/2 \times \Z/2$?
\item Show that this is the unique field with 4 elements up to isomorphism.
  \end{enumerate}

\end{statement}


\end{document}
