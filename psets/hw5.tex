\documentclass[11pt,letterpaper]{article}
\usepackage{amsmath,amsfonts,amsthm}
\setlength{\oddsidemargin}{0in}
\setlength{\evensidemargin}{0in}
\setlength{\textwidth}{6.5in}
\setlength{\topmargin}{-0.4in}
\setlength{\textheight}{8.9in}
\setlength{\parskip}{0.4em}
\newtheorem*{theorem}{Theorem}
\parindent0em
\def\R{\mathbb{R}}
\def\Q{\mathbb{Q}}
\def\N{\mathbb{N}}
\def\Z{\mathbb{Z}}
\def\C{\mathbb{C}}
\def\F{\mathbb{F}}
\raggedbottom

\begin{document}
\begin{center}
{\bf \Large Math 55a Homework 5}\medskip

Due Wednesday October 8, 2025.
\end{center}

\begin{itemize}
\item You are encouraged to discuss the homework problems with 
other students. However, what you hand in should reflect your 
own understanding of the material. You are NOT allowed to copy solutions 
from other students or other sources. Also, please list at the end of the
problem set the sources you consulted and people you worked with on 
this assignment.
%\item Questions marked * may be on the harder side.
\end{itemize}

{\bf Material covered:} Generalized eigenvectors, nilpotent operators, Jordan normal form;
categories and functors; bilinear forms.
(Axler chapter 8; 
Artin \S 4.4-4.7 and 8.1, 8.2, 8.4; handout on categories and functors.)
\bigskip


{\bf 1.} 

(a)
(b) Show that $I+T$ is invertible (where $I$ is the identity operator),
and give a formula for its inverse.
\medskip


{\bf 2.}

%\medskip

{\bf 3.}\medskip

{\bf 4.} Let $V$ be an $n$-dimensional vector space over a field $k$,
with $\ch(k)\neq 2$.
The set of bilinear forms $V\times V\to k$ has the structure of a vector space $B(V)$,
of dimension $n^2$. We say that a bilinear form is {\em symmetric} if $b(v,w)=b(w,v)$ for
all $v,w\in V$, {\em skew-symmetric} if $b(v,w)=-b(w,v)$ for all
$v,w\in V$.

(a) Show that the subset $B_{symm}(V)\subset B(V)$ of symmetric bilinear
forms is a subspace of $B(V)$, and calculate its dimension.

(b) Similarly, show that the subset $B_{skew}(V)\subset B(V)$ of
skew-symmetric bilinear forms is a subspace of $B(V)$ and calculate its
dimension.

(c) Show that $B(V)=B_{symm}(V)\oplus B_{skew}(V)$. Does this remain true if
$\ch(k)=2$?
\medskip

{\bf 5.} Let $V$ be an $n$-dimensional vector space over a field $k$,
with $\ch(k)\neq 2$. 

(a) How should one define the notion of a {\em trilinear form} $t:V\times V\times V\to k$?
Show that the set $T(V)$ of trilinear forms on $V$ can be given the
structure of a vector space, and calculate its dimension.

(b) Say a trilinear form $t:V\times V\times V\to k$ is {\em symmetric} if
the value $t(u,v,w)$ is unchanged if we permute the variables, and denote
by $T_{symm}(V)\subset T(V)$ the subset of symmetric trilinear forms.
Show that $T_{symm}(V)$ is a subspace of $T(V)$, and calculate its
dimension.

(c) Say a trilinear form $t:V\times V\times V\to k$ is {\em skew-symmetric}
(or {\em alternating}) if permuting the variables multiplies the value 
$t(u,v,w)$ by $\pm 1$ according to the sign of the permutation. Show that
the subset $T_{skew}(V)\subset T(V)$ of skew-symmetric trilinear forms
is a subspace of $T(V)$, and calculate its dimension.

(d) Show that, if $n\geq 2$, then we do {\em not} have $T(V)=T_{symm}(V)
\oplus T_{skew}(V)$.  (The reason for this will become clearer when we
discuss the representation theory of the symmetric group.)
\medskip

{\bf 6.} Let $V$ be a finite-dimensional vector space over a field $k$, and
$b:V\times V\to k$ a non-degenerate bilinear form. A subspace
$\Lambda\subset V$ is called {\em isotropic} for $b$ if $b(v,w)=0$ $\forall
v,w\in\Lambda$. Show that, if $\Lambda$ is isotropic, then
$\dim(\Lambda)\leq \frac12 \dim(V)$.
\medskip


{\bf 7.} Let $S_1,\dots,S_m$ be subsets of $\{1,\dots,n\}$ such that
$S_i$ contains an odd number of elements for each $1\leq i\leq m$, 
and $S_i\cap S_j$ contains an even number of elements for each $1\leq
i<j\leq m$. Show that $m\leq n$.

(Hint: construct vectors $v_1,\dots,v_m$ in $(\F_2)^n$, and use the
standard dot product (mod 2) to show they are linearly independent). 
\medskip

{\bf 8.} How long did this assignment take you?  How hard was it?
What resources did you use, and how much help did you need?
(Remember to list the students you collaborated with on this assignment.)
Did you have any prior experience with this material?

\end{document}
